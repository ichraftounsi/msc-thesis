% http://www.swen.uwaterloo.ca/~knaik/ByChapters.html

\documentclass[citeauthoryear]{llncs}
\usepackage{url}
\usepackage{cite}
\usepackage{listings}
\usepackage[utf8]{inputenc}
\usepackage{graphicx}
\usepackage{float}
\usepackage{verbatim}
\usepackage{eurosym}


%\title{Assessing Modeling Languages, metrics and tools}
\title{Study on Test Case Generation Tools}

%Exemplo para adição dos autores
\author{Ulisses Araújo Costa}

\institute{Department of Informatics, University of Minho\\ Campus de Gualtar, 4710-057 Braga, Portugal\\
\email{ulissesmonhecosta@gmail.com}
}

\newcommand{\xunit}{\textit{xUnit}}

\begin{document}
\maketitle
% Software Testing Techniques:
% https://docs.google.com/viewer?a=v&q=cache:blahkDj0Q3QJ:www.cs.cmu.edu/~luluo/Courses/17939Report.pdf+There+are+two+major+forms+of+data+%EF%AC%82ow+testing:&hl=pt-PT&gl=pt&pid=bl&srcid=ADGEEShRSdMyUxeL-7AKWQG2D9wpjOa4orrW9o3F0ntehrAK-w2jq1iaQ16jPWiar28zrG46Etox4qQ8C0kOmDz06pc9r03CPjmWo_voTaZ4VL8db9-2911bQbF7cx7cWeI9RqVaQJqd&sig=AHIEtbSnC3GcvJWzIDjecPhjHlMoqFFpqg
\begin{abstract}
ATCG (Automated Test Case Generation) are tools that give support for creating test cases and at the same time ensure test case coverage methodically.
The main goal of this tools is extract information from the program on how to generate executable test cases.\\
Using  manual written tests is tedious, time consuming and error-prone. Lots of functions/methods need full code coverage and this technique leaves
to incomplete test suites and is hard to create tests that cover specific code paths potentially leaving many hidden bugs.
Besides that, software is not a static artifact and is constantly evolving, so a test generation technique could be a more suitable mechanism in the development process.\\
There are many approaches trying to tackle this problem and therefore many tools were developed. Here I will present a study on the most recent tools that uses:
Model-based testing, Specification-based testing, Constraint-based generation, Random generation and Grammar-based generation for the most used languages - C, Java and C\#.
\end{abstract}

\section{Introduction}
Since ever, industry use methods to discover problems in early stages of development process and improve
the quality of their product and software industry is not an exception. Miller\cite{miller} describe the utility
of software testing as:

\begin{quotation}
The general aim of testing is to affirm the quality of software systems by systematically
exercising the software in carefully controlled circumstances.
\end{quotation}

In the most recent period of software history the integration of software testing as a important step in the process of
software development opened up to the appearance of \xunit\cite{xunit} tools and Agile software development.
Also, companies start to use manual written tests as a part of their software development processes.\\
Using  manual written tests is tedious, time consuming and error-prone. Lots of functions/methods need full code coverage and this technique leaves
to incomplete test suites and is hard to create tests that cover specific code paths potentially leaving many hidden bugs. Many times a supervision leaded by the developer
is needed to assure the right paths in the code are being tested, specially regarding black-box testing.\\
\indent Nowadays we start to observe a rapid increase in the automatic test generation field, this paper intends to be a study regarding the most studied techniques
and the tools that implement them. We will start to discuss the White-box testing vs. Black-box testing and we will from there to the automatic generation techniques.\\

Two different techniques emerge for different proposes, Structural techniques and Functional techniques,
White-box testing and Black-box\cite{black} testing respectively.

\subsection{White-box testing}
In White-box testing the tester needs to understand the internals of the code to be able to write tests for it.
The goal of selecting test cases that tests specific parts of the code is to cause the execution of specific spots in the software, such as statements, branches or
paths. The expected results are evaluated on a set of coverage criteria. Examples of coverage criteria include path coverage, branch coverage, and data-flow coverage.
%\begin{description}
% Code Coverage Analysis:
% http://www.bullseye.com/coverage.html
% Software Testing and Quality Assurance - Control Flow:
% https://docs.google.com/viewer?a=v&q=cache:6QvcPxmUFyUJ:www.swen.uwaterloo.ca/~knaik/MYBOOK1/Ch4-ControlFlowTesting.ppt+control+flow+testing&hl=pt-PT&gl=pt&pid=bl&srcid=ADGEESh9osuIe2hmQ-r6YXXHBvwps_E1zdNH3zVghGKe4tIHwN3GUXagUUQmKqnleoE0SiirRrVfC6mV16yaLTecr1z0OoKa1B9MELI7KR5ir8CsZ4DMTam3y86AfecwaXPUL--y1Qg3&sig=AHIEtbRivm5nl9TBdhUyoQYAkQ42z1uQqA
\paragraph{Control Flow based testing} analyzes all the possible paths in the code and write unit tests to cover multiple paths regarding our criteria.
We generate the CFG (Control Flow Graph) of the program, we select different paths regarding our criteria:
Select all paths, Select paths to achieve complete statement coverage\cite{stt,Ntafos:1988:CST:630792.631017},
Select paths to achieve complete branch coverage\cite{Roper1994,stt}
or Select paths to achieve predicate coverage\cite{stt,Ntafos:1988:CST:630792.631017}.
And we generate the test inputs to make any path execute (exist input so the path can be executed, also called feasible path).
To be able to accomplish this we must generate input data that satisfy all the conditions on the path.\\
But we may have a problem, we may find infeasible paths (paths for which there is no input to the system that can cause that path to be executed)
and we today that a general-purpose algorithm for identifying infeasible paths has been proven to be impossible\cite{infeasible} so, full path coverage
is mathematically impossible. Even if we were able to compute all the possible paths.\\
To avoid this problem we may apply a path selection strategy, an heuristic like: Select as many short paths as possible or Choose longer paths.

% Data Flow Testing:
% https://docs.google.com/viewer?a=v&q=cache:XxIB25uhkaoJ:www.cs.swan.ac.uk/~csmarkus/CS339/dissertations/NewM.pdf+Data+flow+testing+papers&hl=pt-PT&gl=pt&pid=bl&srcid=ADGEEShlAvKZrR7h-XK0DgCkyl3AY_A7rpAsNopgLImb9ekkezxf9-I-g4vq74BTi87q5xozI8WJnMKlsu45y-f-aN3LtSKKt5r2zCsplRbbyMkVVq6-ZtbY6CF2u2h8K-mF8-UJbsOe&sig=AHIEtbRRDyhXAUaZ7aIXqRVx4JcgCV61FQ
% Software Testing and Quality Assurance - Data Flow:
% https://docs.google.com/viewer?a=v&q=cache:64LCfXP-LdAJ:www.swen.uwaterloo.ca/~knaik/MYBOOK1/Ch5-DataFlowTesting.ppt+Data+flow+testing&hl=pt-PT&gl=pt&pid=bl&srcid=ADGEESgnbadiCY2oGJ0rJ6Q_aEqO5srxsIeapB7rcE78Hyb7f2gU8N8a9oEqRPzHQ_eaerJoMt3IJmEC2uzhpA8YUVZVB7ehh-farOJU11xIzbsOpKZMQK77ImKc_wF1L2zukfBRBX0R&sig=AHIEtbRFefrgxnAcEDcypTYbwwnOf_YSiQ
\paragraph{Data Flow testing} select test cases according to CFG of software programs. Also looks at the life cycle (creation, usage and destruction)
of a particular piece of data and observes how it is used along the CFG, this ensures that the number of paths are always finite\cite{dataflow}.\\
Data Flow ca be performed at two conceptual levels, static and dynamic data flow testing. With statical data flow we are able to identify potential
defects (aka data flow anomaly) and analyze the source code. On the other hand dynamic data flow involves actual program execution and bears similarity with control flow testing
since we identify paths to execute them and paths are identified based on data flow testing criteria.\\
The increased level of complexity in modern software systems can result in errors within programs, references may be made to variables that don't exist,
or the value of variables may have changed in an unexpected and undesired manner. A more problematic decision in software testing when to stop testing,
and an \textit{adequacy criterion}\cite{Frankl:1988:AFD:53064.53075} notion was created to be able to determine whether a program has been testes "enough".\\
In more detail Data Flow testing define/use testing paths of the program CFG, this paths link variables, represented as nodes\cite{dataflow}.
Associated with this are a set of test coverage metrics, also defined by Sandra Rapps and Elaine Weyuker in \cite{dataflow}.
The metrics are a set of criteria, essentially – allow the tester to select sets of paths through the program, where "the number of paths selected
is always finite, and chosen in a systematic and intelligent manner in order to
help us uncover errors".
%\end{description}

\subsection{Black-box testing}
Opposed to White-box testing, Black-box testing is based on functionality, so the tester observes a system based in their functional contracts and writes the
pairs of inputs and the expected outputs for a given system and this is used from unit testing of single methods/functions to integration testing
of combinations of the methods/functions to system testing of the final system. The positive aspect of this approach is the functional specification-based testing
regardless of the final implementation and it consequent randomness of $input \times output$ pairs, that is very interesting to catch crashes or failures in the system.\\
In fact there a recent tool - csmith\footnote{See more at: http://embed.cs.utah.edu/csmith/} that can generate random C programs and have already proved
the power of stress-testing compilers, static analyzers, and other tools that process C code\cite{Yang:2011:FUB:1993316.1993532}.\\

The downside of this technique, e.g., for a algorithmically non-trivial implementation it could be impossible to give a good test coverage.
Imagine any traversal or search algorithm over a graph, it is very difficult to generate a good coverage of test cases without the knowledge of the algorithm.

\section{Testing Tools Approaches}

% Practical Model-Based Testing: A Tools Approach
% http://www.cs.waikato.ac.nz/research/mbt/
% MODEL BASED TEST GENERATION TOOLS
% http://www.geocities.com/model_based_testing/
% http://www.agedis.de/documents/ModelBasedTestGenerationTools_cs.pdf
% Wikipedia
% http://en.wikipedia.org/wiki/Model-based_testing
% Microsoft Spec# model-based presentation (for dummies)
% https://docs.google.com/viewer?a=v&q=cache:PVdrO7s75ScJ:www.sasqag.org/pastmeetings/harryr.ppt+model+based+testing&hl=pt-PT&gl=pt&pid=bl&srcid=ADGEESjsdNheYbIWtyBYEfgNss2mRQuVcH0Kh8nejgxYwFMSSoPDuEgtZbxdchxcNXENujrMXaIeDr4Q17lpB1Kk8073iuk76o6G66sppEYAq8wGbUKnNtbOx4etHlT_P2boklZ_DwiO&sig=AHIEtbTECVDePxriKz0fNYbhSo56Ktgvww
\subsection{Model-based Testing}
Predicates as specifications (Alloy is related to OCL (Object Constraint Language) for UML)

% Specification Based Testing:
% http://cs.union.edu/~barrv/Testing/Labs/lab9.html
% Video:
% http://www.youtube.com/watch?v=caElFKbceP0
% Program Verification - Automated Test Case Generation, Part I
% https://docs.google.com/viewer?a=v&q=cache:xNINgGyegt8J:www.cs.uiowa.edu/~tinelli/classes/181/Spring08/Notes/10-ATCG-1.pdf+automatic+testcase+generation+-gui+-uml+-web&hl=pt-PT&gl=pt&pid=bl&srcid=ADGEESg65_S-6rwRNXFWJZjChW3x1J4oIldqfY6feMxBYdluY2LkCivOB8fs19Wtg6z1fhTpvrOQYBiIIyUv_3aRnJS3rzZSBjtnuHvtz3fx48xdJW6Jfj66_YvsMcaInpcw3xmO8t0Z&sig=AHIEtbTkPm7QVWO3evlUquwqV8LT-PvtBA
% http://cs.gmu.edu/~offutt/rsrch/spec.html
\subsection{Specification-based Testing}
Specification Based Testing, refers to the process of testing a program based on what its specification says its behavior should be.
In particular, we can develop test cases based on the specification of the program's behavior, without seeing an implementation of the program. So this clearly a
way of Black-box testing.\\
With this technique we can even start the testing phase and development phase in parallel, because we don't need the implementation
to start the developing test cases. The only thing we need is the functional contracts, $input \times output$ sets for each function/method.\\

There are many tools out there that can perform Specification-based Testing approach namely: \textbf{TestEra}\cite{testera}
that can be used to perform automated specification-based testing of
Java programs. This framework requires as input a Java method, a formal specification\footnote{Specifications are first-order logic formulae.}
of the pre and post-conditions of that method, and
a bound that limits the size of the test cases to be generated.\\
With the pre-condition it automatically generates all nonisomorphic test inputs up to the given bound.
It executes the method on each test input, and uses the method postcondition as an oracle to check the correctness of each output. This tool
uses Alloy's SAT system to analyze first-order  formulae. The authors claim that have used TestEra to check several Java programs including an architecture for
dynamic networks, the Alloy-alpha analyzer, a fault-tree analyzer, and methods from the Java Collection Framework.\\ However this paper have some years already and we
don't find the tool available online.



%Consider the following specification for a program:
%Write a program that simulates a pocket calculator. The input is an arithmetic expression that contains only integers and the arithmetic
%operators $+, -, *, /, \%$, and $**$ (exponentiation). Assume that the input is written in infix notation. Report an error if the input contains characters other than those mentioned above. The expression may be as long as 1,000 characters and as short as 3 characters (e.g., 3+2). The program reads input entered at the terminal and prints the expression's value.
%Without writing a program to solve this problem (you don't really know enough C yet to do it), based only on the specification given, and without knowing anything about the ultimate implementation, generate a set of test data that you think would be sufficient to test a program that would be written in accordance with this specification. If you think the specification is incomplete in anyway, state what assumptions you are making about how it should be completed or clarified.
%Remember that a test case consists of the input data and the expected result for that input data.

\subsection{Constraint-based Generation Testing}

\subsection{Random Generation Testing}
In the random testing approach, test inputs are selected randomly from the input domain of the system
Random testing can be summarized as:

– Step 1 : The input domain is identified
– Step 2 : Test inputs are selected independently from the domain
– Step 3 : The system under test is executed on these inputs
• The inputs constitute a random test set
– Step 4 : The results are compared to the system specification.
• The test is a failure if any input leads to incorrect results
• Otherwise it is a success.

Random testing gives us an advantage of easily estimating software reliability from test outcomes.
Test inputs are randomly generated according to an operational
profile, and failure times are recorded The data obtained from random testing can then be used to estimate reliability

Computing expected outcomes becomes difficult, if the inputs are randomly chosen
Therefore, the technique requires good test oracles to ensure the adequate evaluation of test results
A test oracle is a mechanism that verifies the correctness of program outputs
An oracle provides a method to

- generate expected results for the test inputs, and
- compare the expected results with the actual results of execution of the Implementation Under Test (IUT)

Four common types of oracles are as follows:

- Perfect oracle
- Gold standard oracle
- Parametric oracle
- Statistical oracle

\subsubsection{Adaptive Random Testing}
In adaptive random testing the test inputs are selected from the
randomly generated set in such a way that these are evenly spread
over the entire input domain
The goal is to select a small number of test inputs to detect the first
failure
A number of random test inputs are generated, then the “best” one
among them is selected
We need to make sure the selected new test input should not be too
close to any of the previously selected ones
That is, try to distribute the selected test inputs as spaced out as
possible



\subsection{Grammar-based Generation Testing}

\subsection{Combinatorial Selection Testing}

\subsection{Pex}
O Microsoft Pex vem com um outro pacote que se chama Moles ajuda a escrever unit tests, mas a tarefa entediosa de escrever os testes especificos permanece. 
O Microsoft Pex é uma ferramenta de witebox testing (ou seja, ele analiza o código para produzir os testes) ajuda a compreender o comportamento de entrada/saída do código, encontrar entradas que fazem com que o código em teste dê problemas (faça throws de uma excepção, ...) , e explorar unit tests com parâmetros para verificar se o código implementa a funcionalidade desejada para todas as entradas.
Quando se executa o Pex este gera casos de teste, analisando o código. Para cada declaração no código, o Pex acabará por tentar criar uma entrada de teste que irá atingir esse pedaço de código.
O Pex faz uma análise de caso para cada ramo condicional no código, por exemplo: as declarações, todas as operações que podem lançar exceções, if statements, asserções e contractos (Contracts do .net).

Assim sendo o número de testes que o Pex irá gerar será o número de caminhos possíveis no código.

O Pex pode ser usado num projecto, classe ou até método (o que faz com que seja uma ferramenta muito versátil e comoda de usar). Depois de analizar, este apresenta no "Pex Exploration Results" os pares de (input, output) que obteve para cada métodos e aida dá um resumo da percentagem de cobertura dos testes que gerou.

Experimentei o Pex num projecto C\# de tamanho consideravel (35+ classes), que usa ficheiros XML, ZIP e tem uma GUI em WindowsForms e isto foi os resultados que obtive ou o que me apercebi que ele faz:
- Gera um ficheiro de teste para cada classe publica
- O Pex pode gerar inputs para testes unitários parametrizados , analizando as branch conditions no código que está a ser testado.
- O Pex usa um constraint solver para automaticamente gerar inputs relevantes sobre o códgio que esta a ser testado (ele consegue usar os vários construtores de uma classe).
- Quando um teste gerado falha, o Pex tenta sugerir como o código do programa para que o teste passe (isto acontece muitas vezes com faltas de verificação de nulls nos parametros de entrada).

Conclusão: O Pex é uma ferramenta que já está a ser desenvolvida há 3 anos pela msft e parece estar muito robusta. Ésta ferramenta tenta faer o máximo de trabalho possível e depois deixa na mão do developer um conjunto de ajustes a fazer (no estilo point-and-click) que permite facilmente escolher e gerar o caso de teste que queremos que fique escrito em código.
Num dia consegui testar uma parte considerável da minha aplicação (toda a parte de business practicamente).


%%Conclusão
%%Biblio
\bibliographystyle{alpha}
\bibliography{state}

\end{document} 
