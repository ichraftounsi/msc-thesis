\chapter{Testing}
\minitoc
In this chapter are discussed and contextualized historically the \ac{ECSS} Standards by which the space industry is governed (Section \ref{sec:historyspace} and Section \ref{sec:doc-stand}).
The importance of this Standards is addressed regarding the security of the software products for space purposes, it is also
shown the classification of \ac{ECSS} software criticality (Section \ref{sec:stand}) and starting therefrom are shown testing techniques that 
respond to such high requirements, with the aim to analyze the code coverage (Section \ref{sec:ccoverage}).
\new{
Related to space Standards it is also shown the most reliable Standard for airborn software components -- The DO-178/ED-12 from \ac{EUROCAE} and \ac{RTCA} (Section \ref{sec:airborne}).
Also it will be shown the most recent version of this Standard and the relationship between it and \ac{ECSS} software Standards (Section \ref{sec:new-stand}).\\
The term "safety" denotes the property of a system that it will not endanger human life or the environment and
the term "safety-critical" is used to characterize a
system that is intended to achieve, on its own, the necessary level of safety integrity for the implementation of the required safety functions.
So, a safety-critical software is an artifact that if
fails can cause loss of human life or have other catastrophic consequences.\\
}
In a second part of this chapter the main differences between \ac{WB} and \ac{BB} testing are presented and
some testing tools approaches are described considering automatic generation testing techniques, and some tools for each technique namely:
\ac{SBGT}, \ac{CBGT}, \ac{GBGT} and
\ac{RBGT}.
\section{ESA and ECSS Standardization in Space Industry}
Software products for space industry tends to be a part of a network composed by several systems. This space systems include manned and unmanned spacecraft,
launchers, payloads, experiments and their associated ground equipment and facilities\cite{Mattiello-FranciscoSanAmbJogCos:2007:BrSoIn}.
These kind of projects are generally  expensive, demand considerable amount of time to be completed and have different companies and countries participating.\\
Trying to get a Standard way to follow all rules to deliver the components with the same degree of quality and reliability is extremely important.\\

Software development for Space, particularly in Europe, follows very specific rules on international Standards. These Standards not only define how a software project should
evolve but also how every component developed in the field of Space Industry should be done, tested, documented, deployed and maintained.
This include not only software but many more artefacts like: Electrical, Mechanical and Material components, etc.\\
In 1994, \ac{ECSS} was established to develop a single set of consistent Space Standards recognized and accepted for use by the entire European Space Community.
\ac{ECSS} is supported by \ac{ESA} Council deliberations since the foundation and has developed, through
a partnership between \ac{ESA}, National Space Agencies and European Industry, the management and implementation of European Space Projects.\\
The main objectives of \ac{ECSS} are to increase the effectiveness of all Space programmes in Europe through the application of a
single, integrated set of Standards and Requirements from which all generic requirements of future Space projects can be derived and
improve the competitiveness of the European Space industry.
This is particularly important to formalize an unambiguous communication in order to facilitate the interaction between project partners,
create a set of legally binding documents, reduce risks, guarantee interface compatibility and
improve the quality and safety of Space projects and products.\\
In Section \ref{sec:historyspace} we will see the history of Space Industry Standards and the need for them.

\subsection{History of Software Development Standards for Space Industry}\label{sec:historyspace}
\ac{ESA} Software Engineering Standards originate around 1975. At
that time the \ac{ESA} was working on a number of ambitious projects, on relatively new subjects, involving the development of
large amounts of code. The project teams often consisted of technically excellent engineers, but most of them were not used to work in projects with
rigorous cost and a strict schedule. There was very little project oriented discipline\cite{esa-bulletin-90}.\\

One of the projects had a high level of criticality (as we will see later on Table \ref{tab:severity}) and the
development activity was done at \ac{ESOC}, namely to support the mission, the launch of which was imminent.
The project began delaying and there were some major hardware, operating system problems. The team was waiting for improvements in the hardware
and eventually to be able to continue the work, they reduced the amount of requirements to only
those highly essential for the launch of the satellite. The problem was faced when they realize that no one knew the exact system requirements.
\begin{quotation}
There were no written requirements — at best, some could partly be
retrieved from minutes of meetings. All the rest were in the minds of the project engineers. How
could cost and schedule be guaranteed under these circumstances? This was a project manager's
worst nightmare!\cite{esa-bulletin-90}
\end{quotation}

The team manager decided to stop any further development. All software engineers write down their understanding of the requirements.
As they had a written list of requirements, they were now able to select only
those requirements needed for the launch. The remaining requirements were implemented after
the satellite was launched. In this case a disaster was prevented from happening\cite{Zwartjes05anagile}.
The \ac{BSSC} was subsequently founded in order to \ac{ESA} never end up in an identical situation. This was the seed from
which the \ac{BSSC}, and consequently, the \ac{ESA} Software Engineering Standards was funded and prospered.\\
\ac{ESA}'s \ac{BSSC} was formally established in May 1977 ant it was defined that \ac{BSSC} responsibilities were to:
\begin{itemize}
\item Establish and maintain engineering Standards for the procurement, development and maintenance of software.
\item Verify that the Standards thus established are made applicable to software
development activities at \ac{ESA}. The \ac{BSSC} verifies the application of the PSS-05-0\cite{pss-05-0} Standards through a procedure involving the
scrutiny of contracts with a software content.
These rather limited controls have succeeded remarkably well in helping to instill a software Standards "culture" within the organisation.
\item Act as liaison with \ac{ESA} Legal Affairs department concerning the safeguarding of intellectual
property rights relating to software.
\end{itemize}
PSS-05-0 was a very successful Standard and \ac{BSSC} has played an important role in promoting it, both through its various
publications and through its responsibility in monitoring its application. PSS-05-0 has served to create a common software engineering culture in \ac{ESA}\cite{esa-bulletin-90}.\\
In June 1994, \ac{ESA} Council confirmed the willing to transfer the present PSS system of \ac{ESA} space Standards to a new system
of Standards, today known as \ac{ECSS}, in order to expand the Standards 
to other European national space agencies and European space industry in general.
The \ac{ECSS} system of Standards covers a bigger area, as we explain before: management, product assurance and engineering Standards, specifically for space projects.
The \ac{ECSS} Standard for software engineering, ECSS-E-40\cite{ecss-e-st-40c}, is a high-level requirements-oriented Standard, in effect a tailoring of ISO/IEC 12207 for space projects.
Unlike the old PSS-05-0, it reflects the specifics of space projects, for example the relationship between the space system development cycle and the software development cycle\cite{esa-bulletin-90}.

\subsection{Document Standards}\label{sec:doc-stand}
ECSS-S-ST-00\cite{ecss-s-st-00c} document is the top level user document of \ac{ECSS}.
It gives a general introduction into \ac{ECSS} and the correct use of \ac{ECSS} Documents in space
programmes and projects.
Its purpose is to provide users with an overview of the \ac{ECSS} System, together
with an introduction to the three branches of applicability, (that will be explained later on this Section) and to the disciplines
covered by the set of \ac{ECSS} Standards and the processes involved in generating
and using these Standards.
In this document there is also an introduction into space programmes, space projects actors and their
customer-supplier relationships. The application of this System for space projects in the customer-supplier
chain is also explained and a practical tailoring method is described together with
methods for collecting and processing user feedback.
Finally the top-level requirements are defined for implementation of the \ac{ECSS}
system in space projects/programmes\cite{ecss-s-st-00c}.\\

Under this document there are three parallel branches, the branche dedicated to project management,
engineering, and product assurance.
Project management branch documents have an "M" prefix, engineering Standards have an "E" prefix, and product assurance Standards have a "Q" prefix.
Within each branch, disciplines and corresponding requirements are covered by dedicated Standards.
Standards are normative document written specifically for direct use in invitation to tender and 
business agreements for implementing space related activities. Its content 
is strictly limited to the statement of verifiable customer requirements, supported 
by the minimum descriptive text necessary for understanding their context.

\ac{ECSS} Disciplines can also be supported by \ac{HB} and \ac{TM} as necessary\cite{ecss-s-st-00c}.
\ac{HB} are non-normative document providing orientation, 
guidelines, technical data, advice or recommendations, which contains 
information about how to implement space related activities. There are two types of \ac{HB}'s: Guidelines and good practices and Collection of data.
On the other side, \ac{TM} are non-normative documents providing useful information to the space community 
on a specific subject, prepared to record and present non-normative data, which 
are not relevant for a Standard or for a \ac{HB}, or not yet mature to be published as \ac{HB} or Standard.

In order to clear understand the disciplines addressed by the \ac{ECSS} Standards system,
is shown in Figure \ref{fig:ecssdocsstruct} the Standard document by \ac{ECSS} where it is possible to clearly
see the three separated branches (M, E and Q)\cite{ecss-s-st-00c}.
It is also possible to see Standards from other \ac{SDO} that are integrated into \ac{ECSS} new Standards.

\def\a{\save[].[dddddd]!C="a"*[F--]\frm{}\restore}
\def\b{\save[].[dddddddd]!C="b"*[F--]\frm{}\restore}

\def\h{\save[].[rr]!C="h"*+<7pt>+[F:<100pt>]\frm{}\restore}

\ifx\bw\undefined
\begin{figure}[!ht]
\footnotesize
\begin{displaymath}
\xymatrix@=.28cm{
  & *+[F-]{\txt{ECSS-S-ST-00\cite{ecss-s-st-00c} - ECSS System}} \ar@{->}[d] \ar@{->}[dl] \ar@{->}[dr] & \\
	\h\a{\txt{Space project management\\(management Standards)}}
  & \b{\txt{Space product assurance\\(product assurance Standards)}}
  & \b{\txt{Space engineering\\(engineering Standards)}} & *+[F]{\txt{External\\Standards\\adopted by\\ECSS}}\ar@/{}_{3pc}/@{->}[];"h"+<3.1pt,28.3pt>+\\
	*+[F**:red]{\txt{M-10 - Project planning\\and implementation}}
  & *+[F**:blue][white]{\txt{Q-10 - Product\\assurance management}}
  & *+[F**:green]{\txt{E-10 - System\\engineering}}\\
	*+[F**:red]{\txt{M-40 - Configuration and\\information management}}
  & *+[F**:blue][white]{\txt{Q-20 - Quality assurance}}
  & *+[F**:green]{\txt{E-20 - Electrical and\\optical engineering}}\\
	*+[F**:red]{\txt{M-60 - Cost and schedule\\management}}
  & *+[F**:blue][white]{\txt{Q-30 - Dependability}}
  & *+[F**:green]{\txt{E-30 - Mechanical\\engineering}}\\
	*+[F**:red]{\txt{M-70 - Integrated logistic\\support}}
  & *+[F**:blue][white]{\txt{Q-40 - Safety}}
  & *+[F**:green]{\txt{E-40\cite{ecss-e-st-40c} - Software\\engineering}}\\
	*+[F**:red]{\txt{M-80 - Risk management}}
  & *+[F**:blue][white]{\txt{Q-60 - EEE components}}
  & *+[F**:green]{\txt{E-50 - Communications}}\\
  & *+[F**:blue][white]{\txt{Q-70 - Materials, mechanical\\parts and processes}}
  & *+[F**:green]{\txt{E-60 - Control\\engineering}}\\
  & *+[F**:blue][white]{\txt{Q-80\cite{ecss-q-st-80c} - Software\\product assurance}}
  & *+[F**:green]{\txt{E-70 - Ground systems\\and operations}}\\
  & & &
}
\end{displaymath}
    \caption{ECSS System}\label{fig:ecssdocsstruct}
\end{figure}

%documneto que define o sistema ecss\cite{ecss-s-st-00c}
%documento de qualidade\cite{ecss-q-st-80c}
%ulisses paper\cite{costa_et_al:OASIcs:2012:3523}
\else
\begin{figure}[!ht]
\footnotesize
\begin{displaymath}
\xymatrix@=.28cm{
  & *+[F-]{\txt{ECSS-S-ST-00\cite{ecss-s-st-00c} - ECSS System}} \ar@{->}[d] \ar@{->}[dl] \ar@{->}[dr] & \\
	\h\a{\txt{Space project management\\(management Standards)}}
  & \b{\txt{Space product assurance\\(product assurance Standards)}}
  & \b{\txt{Space engineering\\(engineering Standards)}} & *+[F]{\txt{External\\Standards\\adopted by\\ECSS}}\ar@/{}_{3pc}/@{->}[];"h"+<3.1pt,28.3pt>+\\
	*+[F]{\txt{M-10 - Project planning\\and implementation}}
  & *+[F]{\txt{Q-10 - Product\\assurance management}}
  & *+[F]{\txt{E-10 - System\\engineering}}\\
	*+[F]{\txt{M-40 - Configuration and\\information management}}
  & *+[F]{\txt{Q-20 - Quality assurance}}
  & *+[F]{\txt{E-20 - Electrical and\\optical engineering}}\\
	*+[F]{\txt{M-60 - Cost and schedule\\management}}
  & *+[F]{\txt{Q-30 - Dependability}}
  & *+[F]{\txt{E-30 - Mechanical\\engineering}}\\
	*+[F]{\txt{M-70 - Integrated logistic\\support}}
  & *+[F]{\txt{Q-40 - Safety}}
  & *+[F]{\txt{E-40\cite{ecss-e-st-40c} - Software\\engineering}}\\
	*+[F]{\txt{M-80 - Risk management}}
  & *+[F]{\txt{Q-60 - EEE components}}
  & *+[F]{\txt{E-50 - Communications}}\\
  & *+[F]{\txt{Q-70 - Materials, mechanical\\parts and processes}}
  & *+[F]{\txt{E-60 - Control\\engineering}}\\
  & *+[F]{\txt{Q-80\cite{ecss-q-st-80c} - Software\\product assurance}}
  & *+[F]{\txt{E-70 - Ground systems\\and operations}}\\
  & & &
}
\end{displaymath}
    \caption{ECSS System}\label{fig:ecssdocsstruct}
\end{figure}
\fi

The application of this Standards regarding space related projects is not a strict imposition, they enforce
tailoring to the technical, cost, schedule, programmatic and economic characteristics of the space programmes and projects.
For a specific project, Standards are tailored according to all the traits of the project: the type and 
phase of the project, the acceptable risks, the project complexity, cost, etc.
So, tailoring is the process by which individual requirements are evaluated and made applicable to a specific project.
The available list of Standards and their content is analysed in order to select the set of Standards to be made applicable to the project, taking 
into account its specificity and constrains and identify those requirements in the individual Standards which are not 
applicable to the project, modify or exceptionally add requirement for specific needs\cite{ecss-s-st-00c}.
Tailoring is an important task in order to deliver a suitable product or service considering the costumer needs.

\subsection{ECSS-Q-80 and Testing}\label{sec:stand}
Software development projects tend to be specific depending on the application domain and 
space software projects are not an exception. These projects tend to have particular dynamics than software projects
in other domains because space software development pose additional challenges due to its inherent
requirement that the end products should be highly dependable.
The criticality factor of those systems (sometimes life critical) tend to have different more stringent requirements and approaches 
compared to less critical software projects, so
the existing software engineering Standards are not enough to cope with these challenges, this is why \ac{ECSS} has developed a set of Standards for European
space projects\cite{Ahmad2010AgileECSS,10.1109/SESS.1997.595952}.
This is why \ac{ECSS} have a particular E branch Standard completely dedicated to "Software Product Assurance" --- The ECSS-Q-80\cite{ecss-q-st-80c}.\\

The fundamental principle of these Standard is to facilitate the
customer-supplier relationship assumed for all software developments, at all levels.
They have contributed to these objectives by providing a set of requirements to be
met throughout the system lifetime which involves the development and
maintenance of space application software. Such requirements deal with quality
management, lifecycle activities and process definition, and quality
characteristics of the software product\cite{Mattiello-FranciscoSanAmbJogCos:2007:BrSoIn}.
Therefore the main objective of this Standard is to provide the customer with adequate confidence about the software product.\\

In order to improve greater security in the code that is produced for space industry, the Standard ECSS-Q-80\cite{ecss-q-st-80c} defines
some levels of criticality of the software, depending on their use in the mission.
This categorization is separate into 4 levels (A,B,C and D) regarding what can happen if the software fails or if it is not executed.
In the following listing we can see the classification:

\begin{description}
\item[Level A] Software that if not executed, or not correctly executed, or if anomalous behaviour can cause or contribute to a system failure resulting in \textbf{Catastrophic consequences};
\item[Level B] Software that if not executed, or not correctly executed, or if anomalous behaviour can cause or contribute to a system failure resulting in \textbf{Critical consequences};
\item[Level C] Software that if not executed, or not correctly executed, or if anomalous behaviour can cause or contribute to a system failure resulting in \textbf{Major consequences};
\item[Level D] Software that if not executed, or not correctly executed, or if anomalous behaviour can cause or contribute to a system failure resulting in \textbf{Minor or Negligible consequences}.
\end{description}

Each Severity level is intrinsically connected to a level of safety into the mission. Catastrophic corresponds to
Loss of life, life-threatening or permanently disabling injury or occupational illness; Loss of system; 
Loss of an interfacing manned flight system; Loss of launch site facilities; and Severe detrimental environmental effects. The result
of having a software component fail within this level leaves to major failures propagation in the mission\cite{ecss-q-st-30c}.\\
The Critical level corresponds to Temporarily disabling but not life-threatening injury, or temporary occupational illness;
Major damage to interfacing flight system; Major damage to ground facilities; Major damage to public or private property;
and Major detrimental environmental effects. The result of having a software component fail this level leaves to loss of mission.\\
The other two levels: Major and Minor leaves to Major mission degradation and Minor mission degradation with no or negligible safety results in the mission.
In Table \ref{tab:severity} we have the \ac{ECSS} Severity Classification of consequences information summarized acording to \cite{ecss-q-st-40c}.
According to the Standard
when several categories can be applied to the system or system component, the highest severity takes priority.
\begin{table}[!ht]
\centering
\noindent \begin{tabular}{|c|c|m{4cm}|m{5cm}|}\hline
\textbf{Severity} & \textbf{Level} & \textbf{Dependability (from \cite{ecss-q-st-30c})} & \textbf{Safety}\\\hline\hline
\multirow{5}{*}{\textbf{Catastrophic}} & \multirow{5}{*}{$1$} & \multirow{5}{*}{Failures propagation} & 
Loss of life, life-threatening or permanently disabling injury or occupational illness;\\\cline{4-4}
& & & Loss of system;  \\ \cline{4-4}
& & & Loss of an interfacing manned flight system; \\ \cline{4-4}
& & & Loss of launch site facilities;  \\ \cline{4-4}
& & & Severe detrimental environmental effects.  \\\hline\hline
\multirow{5}{*}{\textbf{Critical}} & \multirow{5}{*}{$2$} & \multirow{5}{*}{Loss of mission} & 
Temporarily disabling but not life-threatening injury, or temporary occupational illness;\\\cline{4-4}
& & & Major damage to interfacing flight system;  \\ \cline{4-4}
& & & Major damage to ground facilities; \\ \cline{4-4}
& & & Major damage to public or private property;  \\ \cline{4-4}
& & & Major detrimental environmental effects.  \\\hline\hline
\textbf{Major} & $3$ & Major mission degradation & --\\\hline\hline
\textbf{Minor} & $3$ & Minor mission degradation or any other effects & --\\\hline
\end{tabular}
\caption{ECSS Severity Classification of consequences}\label{tab:severity}
\end{table}

To ensure more security about the source code, regarding the severity level, \ac{ECSS} Standards, namely \cite{ecss-q-st-40c}
consider the application of formal tools to verify the source code. But at the same time, a major emphasis to software testing.
\ac{ECSS} divides the the test effort regarding the criticality of the source code.
If the source code is classified as Level A then the tests for that piece of software should meet 100\% in
three different categories: Source code statement coverage, Source code decision coverage and Source code modified condition and decision coverage.\\
Statement coverage measure the part of the program within which every executable source code statement has been invoked at least once.
Decision coverage measure of the part of the program within which every point of entry and exit
is invoked at least once and every decision has taken TRUE and FALSE
at least once (this includes statement coverage).
Modified condition and decision coverage measure of the part of the program within which every point of entry and exit
has been invoked at least once, every decision in the program has taken TRUE and FALSE values at least once, and each condition in a decision has been
shown to independently affect that decision's outcome. A condition is shown to independently affect a
decision's outcome by varying that condition while holding fixed all other possible conditions. All of these code coverages will be explained in more detail in Section
\ref{sec:ccoverage}\\
If the source code is classified as Level B then the tests for that piece of software should meet 100\% in the first two categories of testing and should be
agreed with the costumer the value for to meet regarding Source code modified condition and decision coverage.
If the source code is classified as Level C or D then the tests for all the three categories should be agreed with the costumer.
In the following Table \ref{tab:ccoverage} this information is summarized.

\begin{table}[!ht]
\centering
\noindent \begin{tabular}{|m{6cm}|c|c|c|c|}
\hline
\textbf{Code Coverage vs. Criticality} & A & B & C & D \\\hline
Source code statement coverage & 100\% & 100\% & AM & AM \\\hline
Source code decision coverage & 100\% & 100\% & AM & AM \\\hline
Source code modified condition and decision coverage & 100\% & AM & AM & AM \\\hline
\multicolumn{5}{|m{14cm}|}{
NOTE: "AM" means that the value is agreed with the customer and measured for:
unit level testing; integration level testing and validation against the
technical specification and validation against the requirements baseline
as in \cite{ecss-q-st-80c} at clause 6.3.5.2.
}\\\hline
\end{tabular}
\caption{ECSS Code Coverage vs. Criticality}\label{tab:ccoverage}
\end{table}

\new{
\section{DO-178/ED-12 Standard for Airborne Systems}\label{sec:airborne}
Software in avionics has been around since 1970's, but was in the 80's when the cost of computers went down, thanks to the introduction of personal computers,
that the avionics industry start to use software components in airborne systems.
The increased use of Software and Computer systems for safety-critical applications motivated the development in 1980 of
DO-178 by \ac{RTCA}
\footnote{
\ac{RTCA} is a US non-profit organization that develops technical guidance for use by government regulatory authorities and by industry.
The documents it publishes are treated as guidelines, not as requirements.\\
\ac{RTCA}'s objectives are to:
\begin{itemize}
\item Ensure the safety and reliability of airborne systems;
\item Develop minimum operational performance requirements for document-specific systems;
\item Develop guidelines for use by a regulatory authority;
\item Provide administrative and logistics resources that enable teamwork among the world-wide aviation community.
\end{itemize}
}
and ED-35 by \ac{EUROCAE}
\footnote{
\ac{EUROCAE} has now been operating for more than 40 years as a non-profit organization
whose membership exclusively comprises aviation stakeholders made up of Manufacturers,
Services Providers, National and International Aviation Authorities and Users (Airlines, Airports, operators) from Europe.
}
leading to the creation of a single Standard and some years later in 1985 --- The DO-178/ED-12 Standard: The first
common certification criteria for production of avionics software. This document has the title
"Software Considerations in Airborne Systems and Equipment Certification" and became a worldwide basis for software
certification in the aviation industry until nowadays. This Standard is
officially recognized as a \textit{de facto} international Standard by the \ac{ISO} for software certification in the aviation industry.\\
Other Software Standards like MIL-STD-498, MIL-STD-2167A, IEEE/EIA-12207, IEC 61508 deal with certain aspects of
software development covered by DO-178/ED-12\cite{ed-12b} and none of them has been found to provide a complete coverage of Software Development process as DO-178/ED-12 do
\cite{avionics}.

DO-178/ED-12 defines five Software Levels\cite{ed-12b}, such that the requirements of any system can be mapped to one
of these levels:
\begin{description}
\item[Level A] Most critical failure level. Failure at Level A results in catastrophic failure conditions for an aircraft;
\item[Level B] Software at this level contributes to severe-major failure conditions;
\item[Level C] Software at this level contributes to major failure conditions;
\item[Level D] Software at this level contributes to minor failure conditions;
\item[Level E] Software has no effect toward potential failure conditions.
\end{description}
This classification is very similar to the one seen in Section \ref{sec:stand}.
The fact is that this Standard has served as initial motto for ECSS-Q-80\cite{ecss-q-st-80c} and ECSS-E-40\cite{ecss-e-st-40c}, \ac{ECSS} Standards.
In fact \ac{ECSS} software Standards were naturally fetched from many learned lessons in avionics.
Many other definitions in ECSS-Q-80\cite{ecss-q-st-80c} and ECSS-E-40\cite{ecss-e-st-40c} were drawn based on DO-178/ED-12, like the
Severity classification of consequences, Code Coverage vs. Criticality, and many more guidelines.

DO-178/ED-12 has three Software Life cycle Processes where are defined the objectives of each step in the software development as well
the set of activities for meeting the objectives:
\begin{description}
\item[Software Planning Process] where the Standard defines five types of planning data\footnote{\ac{RTCA} and \ac{EUROCAE} avoid referring to this "data" as
"documents" when trying to elaborate on the objective evidence that need to be produced to satisfy the Standard objectives\cite{avionics}, on the other side \ac{ECSS} refers
these evidences as "documents". So we can consider these evidences as documents.} that should include
considerations on the methods, languages, Standards and tools to be used during the development phase. These plans are: Software Aspects of
Certification Plan,
Software Development Plan, Software Verification Plan, Software Configuration Management Plan and Software Quality Assurance Plan.
\item[Software Development Process] where the Standard defines that should be included the requirements, that can be
expressed as high- and low-level requirements\cite{avionics,ed-12b}. The Standard allow the system complexity and the design methodology applied to the system under development
drive the requirements decomposition process. The design, coding and integration artifacts are briefly described since these tend to vary between various development methodologies.
However the Standard defines clearly the expected outputs of each of the processes, the design process yields low-level requirements and software architecture.
The coding process produces the source code, in a high-order language or assembly code. The integration process result in executable code
resident on the target computer. The Standard also predicts that each of these outputs should be verified, assured and configured.
\item[Integral Process] Four processes are defined as integral in DO-178/ED-12 which
mean that they can be overlaid and extended throughout the
software life cycle. They are as follows:
\begin{itemize}
\item Software Verification: Verification is the most important
in DO-178/ED-12 which accounts to over two thirds of the total.
\item Software Configuration Management: The clear definition
of what is to be verified will be more reasonable for the
various outputs of the verification in DO-178/ED-12. This definition
or configuration is the design of the DO-178/ED-12 objectives
for configuration management.
\item Software Quality Assurance: \ac{SQA} objectives provide oversight of the entire DO-178/ED-12
process and require independence at all levels. \ac{SQA} guarantees
detection of plans and Standards, deviated during development
process are recorded, evaluated, tracked and resolved.
\item Certification Liaison Process: As mentioned earlier, the
issues should be identified early for the certification process.
DO-178/ED-12 outlines twenty data items to be produced in a
compliant process where three of these are specific to it and
must be provided to the certifying authority: Plan for Software Aspects of Certification, Software Configuration Index and Software Accomplishment Summary.
\end{itemize}
\end{description}
\subsection{The new Standard DO-178C/ED-12C}\label{sec:new-stand}
The first version of DO-178/ED-12 was DO-178A/ED-12A and was released in 1985 and twenty years later the Standard had a new update -- The DO-178B/ED-12B released in 1992,
ten more years and in 2012 the most recent version of the Standard was released following the existing nomenclature: DO-178C/ED-12C.
The most recent version of the Standard takes into account new technologies and methodologies developed since 1992 (the previous version of the Standard).
These methodologies and Technologies include the \ac{OO} Paradigm, Model based design/automatic code generators and \ac{RT} operating systems.
The new Standard also recognizes that there is more to software verification than testing \textit{per se} and includes Formal methods\cite{Brosgol:2011:DNA:2070337.2070341} and abstract interpretation
and also takes into account the supplementary papers/commentaries on DO-178B.\\
The most recent version includes three supplements: 
\begin{description}
\item[ED-216 Formal Methods] identifies the additions, modifications and substitutions to ED-12C and ED-109A objectives when formal methods are used as part of a software life cycle, and the additional guidance required. It discusses those aspects of airworthiness certification that pertain to the production of software, using formal methods for systems approved using ED-12C;
\item[ED-217 Object-Oriented Technology and related techniques] identifies the additions, modifications and deletions to ED-12C and ED-109A objectives 
when \ac{OO} technology or related techniques are used as part of the software development life cycle and additional guidance is required.
This supplement, in conjunction with ED-12C, is intended to provide a common framework for the evaluation and acceptance of \ac{OO} technology
and related techniques \ac{RT}-based systems;
\item[ED-218 Model-Based development and verification] contains modifications and additions to ED-12C and ED-109A. Objectives, activities, explanatory text and
software life cycle data that should be addressed when model-based development and verification are used as part of the software life cycle.
This includes the artifacts that would be expressed using models and the verification evidence that could be derived from them.
Therefore, this supplement also applies to the models developed in the system process that define software requirements or software architecture.
\end{description}

The benefits of incorporating the \textbf{ED-216 Formal Methods} supplement are important because
the use of mathematical models, that define the system using an unambiguous abstraction syntax (eg state machines, Z notation, set theory, \ac{DSL}'s),
and mathematical analysis, that provides guarantees/proofs of software properties regarding the compliance with requirements,
may be applied at various stages during the software development and can be used as a formal verification technique.
This supplement sees Formal methods as a complement to testing, basically testing shows that functional requirements
are satisfied and sometimes is able to detect errors and Formal methods can increase confidence that no anomalous behavior will occur
(formal methods may found faults that are not detected by testing techniques)\cite{Chelini:2009:WTD:1647420.1647443}.\\
The benefits from incorporating the \textbf{ED-217 Object-Oriented Technology and related techniques} supplement are important because
it assumes that \ac{OO} paradigms brings more safety to software development than other lower level languages once massively used in aerospace.
The Data-centric approach of \ac{OO} languages eases maintenance of many large systems, the use of Model-driven architecture tool (like \ac{UML}) facilitates
the generation of \ac{OO} code\cite{Elliott:2010:OSC:1869542.1869558}. The use of a Standard paradigm allow space industry to use languages used by many programmers such as C++, JAVA or Ada95.
The features that this paradigm brings to space software are very important, namely, the capability of Method overloading, the type conventions, the re-use of
generic templates, exceptions and concurrency, etc.\\
The benefits from incorporating the \textbf{ED-218 Model-Based development and verification} supplement are essentially the motivation to use unambiguous notation and supports the use of
automated code generation tools, namely automatic test generation tools\cite{Chelini:2009:WTD:1647420.1647443}. This supplement acknowledges the need of
high-level requirements from which development model is derived and the ability to demonstrate properties for the model, showing that the property 
is preserved in the source code.\\

As can be seen the Standard ECSS-Q-80\cite{ecss-q-st-80c} in the software quality assurance side and ECSS-E-40\cite{ecss-e-st-40c} in the software development side
give a distinguished emphasis on software validation and verification, however is the Standard DO-178/ED-12, for obvious reasons,
that gives much more emphasis in the verification component. The biggest proof is the most recent embedding of the Formal Methods into the
new version of the Standard DO-178C/ED-12C\cite{ed-12c}, in order to reach where the tests are not enough.
Increasing the use of tools and methodologies from the area of Formal Methods is crucial to prevent software errors and ensure quality and their proper functioning.

\section{Code Coverage}\label{sec:ccoverage}
As we have seen in Sections \ref{sec:stand} and \ref{sec:new-stand}, code coverage is a major concern regarding the software development for space industry,
it is very important to provide a study on this subject
in order to access a possible integration of this feature into the final product produced to \ac{VST}.\\
}
%Brand NEW STUFF
\neww{

\cite{ccoverage}
\cite{mcdc}
\cite{Marick99}
%\subsection{Statement Coverage}
%\subsection{Decision Coverage}
%\subsection{Condition Coverage}
%\subsection{Multiple Condition Coverage}
%\subsection{Condition/Decision Coverage}
%\subsection{Modified Condition/Decision Coverage}
%\subsection{Path Coverage}
%\subsection{Function Coverage}
%\subsection{Call Coverage}
%\subsection{Linear Code Sequence and Jump (LCSAJ) Coverage}
%\subsection{Data Flow Coverage}
%\subsection{Object Code Branch Coverage}
%\subsection{Loop Coverage}
%\subsection{Race Coverage}
%\subsection{Relational Operator Coverage}
%\subsection{Weak Mutation Coverage}
%\subsection{Table Coverage}
\section{Test Categories}
The different categories of software testing enables the detection of faults of the system from different perspectives and in different stages
of software development.
In certain situations, because there are few scenarios or low criticality of the system,
certain categories of testing can be planned together. According to \cite{citeulike:453686} there are five testing categories,
\ac{UT}, Component Testing, Integration Testing, System Testing and Acceptance Testing.
In this section we will explain all of them, with a particular emphasis in \ac{UT} since it is the most elementary testing category
in which is possible to test atomic components of source code (a certain instruction, method/function or class).\\

\ac{UT}s are designed to exercise small parts or units of the system, usually code statements, functions, methods or classes,
determining whether it works correctly and it concentrates on smallest unit of software design, it seeks to identify logical errors
and implementation errors in each module of the software separately.
Almost all programmers perform some level of \ac{UT} on a unitary component or piece of code.
These tests together with the integration tests are essential to create a quality software product\cite{dustin2002effective},
yet they are often overlooked, or are implemented in a superficial way.\\
According to \cite{Hunt:2003:PUT:1197403} it should be done in source code written by a programmer, usually a tester,
who exercises a very small area of the code being tested.\\
\ac{UT}s must check five characteristics: The individual components to make sure that they function properly;
The interface components for ensuring that information is being passed according to the interfaces requirements;
Handling of inconsistent data; Initialization of variables and if the boundary conditions are operating properly.\\
The main motivations to perform \ac{UT} are using a method of testing the elements of a computer program,
in which initial attention is focused on smaller units of the program. Helping in the location, definition and correction of a software problems,
because when an error is located, is known the module and the unit in which it belongs\cite{Myers:2004:AST:983238}.\\
\ac{UT} facilitates isolating bugs because when a problem is found at the unit level, the problem must lie in the unit.
If on the other hand a bug is located, where multiple units are combined then it must be analyzed how those modules interact with each other.
Of course, one can find exceptions, but overall, testing and debugging is much more efficient than test everything immediately\cite{Patton:2000:ST:517489}.\\

If unit tests are carried out in a proper manner, the further phases of tests (as we shall see below) will be more successful.
To perform a structured and repeatable software program with Unit Tests, they should be developed before or in parallel with the development of software.
To be effective and helpful for the development team, \ac{UT} must be considered part of the development project
and tests must be updated along according to the requirements and source code, as the project evolves\cite{dustin2002effective}.
Finding defects while a component is still in development offers a significant savings in time and cost and consequently produces a better quality software product.
The main advantages of use \ac{UT} during software development are\cite{dustin2002effective}:
\begin{itemize}
\item The development must be geared to run the unit tests, in which each requirement will be attending, so the software will be considered
successfully when all the unit tests pass;
\item Will cause the developer to put efforts on meeting the exact problem, and usually it will result in less code and a more direct implementation.
So, using \ac{UT} has a side effect of improving the code and understand their functioning better.
\end{itemize}

In the following listing is shown a brief description for the others test categories:

\begin{description}
\item[Component Testing] The target universe of this type of test is a little larger than \ac{UT},
here it is important to test the components as a whole, but not considering the interaction with other parts of the system;
\item[Integration Testing] The aim of this testing category is to tests the interface between the integrated units.
It is a systematic, for construction of the program structure by performing at the same time, tests to detect errors associated interfaces.
Mainly it tests if units tested individually perform properly when placed together.
\item[System Testing]
The main purpose is to compare the system or program with its original goals.
Here are performed tests considering the whole picture, software elements integrated operating environment.
Usually is performed using \ac{BB} techniques and are executed by a tester.
\item[Acceptance Testing] Usually performed by a small group of end users of the system by simulating routine operations of the system, aiming
verify the behavior of the system is according with the expectations of the end user.
\end{description}

\section{Test Strategies}
There are several ways to test software. There are techniques that were widely used in systems developed in structured languages
that still have great value to the \ac{OO} languages. Despite the development paradigms are completely different,
the main objective of these techniques remains the same: to find flaws in software\cite{Myers:2004:AST:983238}.
The most widely known are the \ac{WB} and \ac{BB}.
These methods complement each other and must be applied together in order to ensure a good quality test.
\subsection{White-box vs Black-box testing}
}
%In this section is discussed the two most common approaches for testing: White-box and Black-box testing.\\
In \ac{WB} testing the tester needs to understand the internals of
the code to be able to write tests for it.
The goal of selecting \ac{TC}s that test specific parts of the code
is to cause the execution of specific spots in the software, such as
statements, branches or paths.
This technique consists in analyzing statically a program, by reading
the program code and using symbolic execution techniques to simulate
abstract program
executions in order to attempt to compute inputs to drive the program
along specific execution paths or branches, without ever executing the
program. Control Flow based testing approach can be useful to analyze all the
possible paths in the code and write unit tests to cover multiple paths.
The \ac{CFG} of the program can be built,
test inputs can be generated to make any path execute regarding a given criterion:
Select all paths;
Select paths to achieve complete statement
coverage\cite{stt,Ntafos:1988:CST:630792.631017};
Select paths to achieve complete branch coverage\cite{Roper1994,stt};
or Select paths to achieve predicate
coverage\cite{stt,Ntafos:1988:CST:630792.631017}.

Data Flow Testing is designed into looking at the life cycle
(creation, usage and destruction) of a particular
piece of data and observe how it is used along the \ac{CFG}, this ensures
that the number of paths is always finite\cite{dataflow}.\\

Opposite to \ac{WBT}, \ac{BBT} is based on
functionality, so the tester observes a system based
on its functional contracts and writes the pairs of inputs and the
expected outputs.
This approach is used for unit testing of single methods/functions,
integration testing
of combinations of the methods/functions, or even final system testing.\\

%This document is organized as follows.
%In Section~\ref{testingapproaches} the important testing approaches in
%use---Specification-based testing and Constraint-based generation---are briefly
%revisited and, for each one,  the most relevant tools are identified.
%In Section~\ref{testingtools} some of the tools referred are
%experimented in order to be compared.
%Our proposal for a test generation system is introduced in
%Section~\ref{proposal}.
%The document is concluded in Section~\ref{sec:Concl}.

\section{Tool Approaches}\label{testingapproaches}
In this Section, a study of the most recent approaches to the development testing tools,
like \ac{SB}, \ac{CB}, \ac{GB} and \ac{RB} tests generation
for the most popular languages - C, JAVA and C\#, will be presented.

\subsection{Specification-based Generation Testing}
\neww{
\ac{SBT} appeared in order to address the testing activities in a systematic manner, this subject is currently under
much discussed in academia and have been gaining interest in the software industry.
According to the philosophy of \ac{SBT}, \ac{TC}s are derived from a system specification\cite{conrad:graph}, so it can be generated systematically from specifications,
with this technique the testing phase and development phase can be started in parallel, since it is not
needed the implementation to start the development of \ac{TC}s. The only thing needed is the functional contracts (The specification, or a model)
and/or oracles\footnote{A test oracle determines whether or not the results of a test execution are correct\cite{Peters95generatinga}.} for each function/method.\\

These specifications serve as input for the definition of \ac{TC}s to test a given system.
If the generation of the test is performed automatically, any change in the system specification will have repercussion in the \ac{TC}s.
However, there are some difficulties in application of \ac{SBT}, as mentioned by\cite{Bertolino04towardsanti-model-based},
namely that exist a gap between the level of abstraction of a model and level of abstraction of \ac{TC}s.
\ac{SBT} are composed of several activities. The main ones are\cite{Lindholm06model-basedtesting} the
\textbf{Construction of the model}: For the definition of \ac{SBT}, various types of models has been used, for example, \ac{UML} diagrams with
\ac{OCL} annotations\cite{Simula.SE.30}, Z specifications\cite{Horcher95improvingsoftware,Stocks:1996:FST:239916.239918},
\ac{UML} Statecharts\cite{Offutt:1999:GTU:1767297.1767341}, \ac{VDM}\cite{Aichernig99automatedblack-box},
\ac{ADL} specifications\cite{Sankar94specifyingand}, or models defined in B Language\cite{Legeard:2002:ABT:647541.730142,Bernard:2004:GTS:1134155.1134156}.
The use of formal specifications leads to a better understanding of the system, as well as providing means for automatic identification of incomplete
or inconsistent specifications so that the other steps of \ac{SBT} have less prone to errors specification.
Since the use of formal specifications tends to diminish ambiguities, that natural languages could create, automatic generation of tests is privileged using this philosophy.
\textbf{Definition of derivation criteria of \ac{TC}s from the model}: Must first establish a technique to be used in the design of the \ac{TC}s,
the techniques may be ac{WBT} (those where the system source code is visible) or \ac{BBT} (where only the specification is considered - the code is not taken into account).\\
The derivation criteria defines the test coverage. For example in \cite{Lindholm06model-basedtesting}, where is generated \ac{TC}s from \ac{FSA},
two types of covering are considered, the first is when the test suite reaches each model state at least once and the second
is when the test suite reaches each state transition at least once.
Obviously, the latter is tightened, creating a greater number of tests, since all the combinations of the two states by two
are achieved at least once. However, in many cases test coverage condition may be infeasible, depending on the model.\\
\textbf{Generation of \ac{TC}s and test suite}: The derivation of test criteria defined in the previous section are applied.
From this interaction \ac{TC}s are generated and grouped into test suites.
\textbf{Adaptation of \ac{TC}s}: 
By being created from an abstract model, generated \ac{TC}s
may have a level of abstraction that is not interesting for system testing.
If this occurs, it is necessary to make an adaptation of this suite to the concrete level of the system being tested.
}
These specifications typically do not consider structurally complex inputs and the existing tools do not generate JUnit \ac{TC}s.
Nowadays there are some commercial tools out there that can perform \ac{SBT} approach, namely:
\begin{description}
\item[Conformiq] is a commercial Tool Suite that generates
human-readable test plans and executable test scripts from Java code, state charts and \ac{UML}\footnote{See more at: \url{http://www.conformiq.com/products.php}}.
\item[MaTeLo] stands for Markov Test Logic and is a commercial tool
that generates test sequences from a collection of states, transitions, classes of equivalence, types, sequences, global variables and test oracles
using their user interface\footnote{See more at: \url{http://www.all4tec.net/index.php/All4tec/matelo-product.html}}.
\item[Smartesting CertifyIt] is a commercial tool that generates \ac{TC}s from a functional model, as \ac{UML}\footnote{See more at: \url{http://www.smartesting.com/index.php/cms/en/product/certify-it}}.
\item[T-Vec] is a commercial tool that generates \ac{TC}s from modeling tools available from T-VEC or third-party vendors\footnote{See more at: \url{http://www.t-vec.com/}}.
\item[Rational Tau] is an \ac{IBM} commercial tool that provides automated error checking, rules-based model checking, and a \ac{SB} explorer using
\ac{UML}\footnote{See more at: \url{http://www-01.ibm.com/software/awdtools/tau/}}.
\end{description}

The relevant ones or the recent open-source ones will be discussed.
\paragraph{Spec Explorer}
This is a Microsoft \ac{SBT} that uses one software modeling languages, the \ac{AsmL}.
This modeling language provides the foundations of the Spec Explorer\footnote{See more at: \url{http://research.microsoft.com/en-us/projects/specexplorer/}} tool
and Spec\# that is a formal language for \ac{API} contracts (influenced by \ac{JML}, \ac{AsmL}, and Eiffel), which extends C\# with constructs for non-null types,
pre-conditions, post-conditions, and object invariants\footnote{See more at: \url{http://research.microsoft.com/en-us/projects/specsharp/}}.
These tool is already available to users and is in a very mature phase.\\
\indent The user of Spec Explorer writes a model of the system and sets the possible values for some properties in his code, furthermore the user also provides a scenario.
These scenarios are simple sets of calls to methods without their parameters (remember that this is Spec Explorer job).
Then Spec Explorer will generate a visual graph where each node represents a state of the system and the arrows represent a call to some method.
It searches throw all possible sequences of methods invocation that do not violate the contracts (pre, pos conditions) and
that are relevant to a user-specified set of test properties. After that we can generate from this visual graphs the unit tests (the arrows) and the
\ac{TC}s (a graph).

\paragraph{JMLUnit}
JMLUnit\cite{Cheon04thejml} is a tool that automates the generation of oracles for JAVA testing classes. This tool
monitors the specified behavior of the method being tested to decide whether the test passed or failed.
This monitoring is done using the formal specification language runtime assertion checker.
The main idea behind these tools is to translate the pre- and post-conditions methods into the code of the testing method.\\
The pre-conditions became the criteria for selecting test inputs, and the post-conditions provided the properties to check for
test results. So, the post-conditions became the test oracles.\\
This tool uses the \ac{JML}\cite{Burdy03anoverview} specification language to annotate JAVA methods code with pre- and post-conditions and
automatically generate JUnit test classes from \ac{JML} specifications.

\paragraph{TestEra}
TestEra\cite{testera} can be used to perform automated \ac{SBT} of
JAVA programs. This framework requires as input a JAVA method, a formal specification\footnote{Specifications are first-order logic formulae.}
of the pre and post-conditions of that method, and a bound that limits the size of the \ac{TC}s to be generated.\\
With the pre-condition it automatically generates all non-isomorphic test inputs up to the given bound.
It executes the method on each test input, and uses the method post-condition as an oracle to check the correctness of each output. This tool
uses Alloy's\footnote{Alloy is a first-order declarative language based on sets and relations. The Alloy Analyzer is a fully
automatic tool that finds instances of Alloy specifications: an instance
assigns values to the sets and relations in the specification such that
all formulae in the specification evaluate to true.} \ac{SAT} system to analyze first-order  formulae.
The authors claim that have used TestEra to check several JAVA programs including an architecture for
dynamic networks, the Alloy-alpha analyzer, a fault-tree analyzer, and methods from the JAVA Collection Framework.

\paragraph{Korat}
Korat\cite{Boyapati02korat:automated} is a mature framework for automated testing structurally complex inputs of JAVA programs.
Given a formal specification for a method, Korat\footnote{See more at: \url{http://korat.sourceforge.net/}} uses the method pre-condition
to automatically generate all (non-isomorphic) \ac{TC}s up to a given small size.
Korat then executes the method on each \ac{TC}, and uses the method post-condition as a test oracle to check the correctness of each output.\\
To be able to generate \ac{TC}s for a method, Korat uses a predicate and a bound on the size of its inputs,
Korat generates all (non-isomorphic) inputs for which the predicate returns $true$.
Korat generates all the possible input spaces regarding the predicate and monitor the predicate's executions to be able to prune large portions of the search space.\\
\indent The writing of a predicate is done using JAVA language and in most cases can be written the first thing that cames to programmer's head to restrict the input space.
But for more complex structures it is better to understand how the matching algorithm work to be able to write a fast verifiable predicate.\\
Unfortunately the test derivation tool using Korat (that also uses \ac{JML}) is not available to the public.

\subsection{Constraint-based Generation Testing}
\ac{CBT}\cite{DeMillo91constraint-basedautomatic} can be used to select \ac{TC}s satisfying specific constraints by
solving a set of constraints over a set of variables. The system is described using constraints and these can be solved by \ac{SAT} solvers.\\
Constraint programming can be combined with symbolic execution, regarding this approach a program is executed symbolically,
collecting data constraints over different paths in the \ac{CFG}, and then solving the constraints and producing \ac{TC}s from there.
There are some tools out there, like:

\begin{description}
\item[Euclide] for verifying safety properties over C code using \ac{ACSL} annotations, CPBPV for program verification.
\item[OSMOSE] a tool that uses concolic execution and path-based techniques over machine code.
\item[GATeL] for Lustre language to generate test sequences\footnote{See more at: \url{http://www-list.cea.fr/labos/gb/LSL/test/gatel/index.html}}.
\end{description}

Here two tools will be explained, one proprietary and other academic.

\paragraph{Pex} Pex\cite{Tillmann:2008:PWB:1792786.1792798} is an automatic \ac{WB} test generation tool for .NET. Starting from a
method that takes parameters, Pex performs path-bounded model-checking
by repeatedly executing the program and solving constraint systems to obtain inputs that will steer the program along different execution paths.
This uses the idea of dynamic symbolic execution\cite{Tillmann06unittests}. Pex uses the theorem prover and
constraint solver Z3\footnote{See more at: \url{http://research.microsoft.com/en-us/um/redmond/projects/z3/}} to reason about the feasibility of execution paths, and
to obtain ground models for constraint systems.\\
Pex came with Moles that helps to generate unit tests. These tools together are able to understand the input (by analyzing branches in the code:
declarations, all exceptions throws operations, if statements, asserts and .net Contracts). With this information Pex uses Z3 constraint solver to
produce new test inputs which exercise diferent program behavior.\\
The result is an automatically generated small test suite which often achieves high code coverage.\\
Pex can be used in a project, class or method (which makes it a very helpful and versatile tool). After the analysis process the "Pex Explorarion Results" shows
the $input \times output$ pairs selected for each \ac{TC} for the method, here it also shows the percentage of the test coverage.

\paragraph{PathCrawler} This is an academic tool based on dynamic and static analysis\cite{Williams05pathcrawler:automatic}, 
it uses constraint logic programming to generate the \ac{TC}s. PathCrawler\footnote{See more at: \url{http://www-list.cea.fr/labos/gb/LSL/test/pathcrawler/index.html}} executes an instrumented function for each function under test
with the generated inputs, it preserves this information to not cover the same path.\\
This tool supports assertions in any point in the code and pre-conditions regarding the input values.

\subsection{Grammar-based Generation Testing}
In this approach inputs to a system under test are defined by a context-free grammar. The language of the grammar contains all possible \ac{TC}s.
Using this approach to describe the syntax of the input to the system under test proves to be very helpful to test
network protocols\cite{tal:syntax-based,kaksonen2001functional} and parsers and compilers\cite{1994-burgess,Burgess_Saidi_1996}.

\paragraph{ASTGen}
ASTGen\cite{Daniel:2007:ATR:1287624.1287651} is a JAVA framework that automates testing of refactoring engines: generation of test inputs
and checking of test outputs. The main technique is an iterative generation of structurally complex test inputs.
ASTGen\footnote{See more at: \url{http://mir.cs.illinois.edu/astgen/}} allows developers to write imperative generators whose executions
produce input programs for refactoring engines. More precisely, ASTGen
offers a library of generic, reusable, and composable generators that produce \ac{AST}.\\
So, ASTGen ensures the production of test inputs instead of the developer produce them. The developer needs to write a generator whose execution
produces thousands of programs with structural properties that are relevant for the specific refactoring being tested. This tool has found
21 bugs in Eclipse and 26 bugs in Netbeans applications.

\subsection{Random-based Generation Testing}
In the \ac{RBT}, test inputs are selected randomly from the input domain of the system.
To have a random testing suite first we must identify the input domain, after that select test inputs independently from the domain,
then the system under test is executed on these inputs, the results are compared to the system specification, an oracle.\\
Random testing gives us an advantage of easily estimating software reliability from test outcomes.
Test inputs are randomly generated according to an operational profile, and failure times are recorded.
The data obtained from random testing can then be used to find bugs or non expected behaviors.\\
\indent The main problem regarding random generation is the problem of the coverage, it is possible that it will not be broad enough. And furthermore it can be
too sparse to actually test specifics parts of the program. Either way, this technique proves to be very effective for testing compilers.

\paragraph{Csmith}
Csmith\cite{Yang:2011:FUB:1993316.1993532} is a \ac{BB} random tests generator that is able to generate C programs
conform to the C99\footnote{See more at: \url{http://www.open-std.org/jtc1/sc22/wg14/www/docs/n1256.pdf}} Standard. This is a very recent tool that already discover
more than 195 bugs in \ac{LLVM} and 79 bugs in \ac{GCC}. With Csmith we are able to generate random programs with unambiguous meanings (undefined behavior or 
unspecified behavior). Does not attempt to generate terminating program, so they use timeouts for long time consuming generated programs.
And the main supported features right now are: Arithmetic, logical, and bit operations on integers, Loops, Conditionals, Function calls, Const and volatile,
Structs and Bitfields, Pointers and arrays, Goto, Break and continue. The generation of code regarding this features can be tuned using the command line program.

\paragraph{QuickCheck for JAVA}
QuickCheck was originally a combinator library for the Haskell\footnote{See more at \url{haskell.org}} programming language\cite{Claessen:2000:QLT:357766.351266}.
Later on QuickCheck philosophy spread to other programming languages like: JAVA, Erlang, Perl, Ruby and JavaScript.\\
QuickCheck works by generating high amounts of data (within the method domain) and checking it against a given property,
it is expected to create a wide range of the input domain, thus increasing the chances of giving more test coverage.

\section{Summary}
summary\ldots
\secendnote
